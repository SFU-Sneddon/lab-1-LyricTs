% Options for packages loaded elsewhere
\PassOptionsToPackage{unicode}{hyperref}
\PassOptionsToPackage{hyphens}{url}
%
\documentclass[
]{article}
\usepackage{amsmath,amssymb}
\usepackage{lmodern}
\usepackage{iftex}
\ifPDFTeX
  \usepackage[T1]{fontenc}
  \usepackage[utf8]{inputenc}
  \usepackage{textcomp} % provide euro and other symbols
\else % if luatex or xetex
  \usepackage{unicode-math}
  \defaultfontfeatures{Scale=MatchLowercase}
  \defaultfontfeatures[\rmfamily]{Ligatures=TeX,Scale=1}
\fi
% Use upquote if available, for straight quotes in verbatim environments
\IfFileExists{upquote.sty}{\usepackage{upquote}}{}
\IfFileExists{microtype.sty}{% use microtype if available
  \usepackage[]{microtype}
  \UseMicrotypeSet[protrusion]{basicmath} % disable protrusion for tt fonts
}{}
\makeatletter
\@ifundefined{KOMAClassName}{% if non-KOMA class
  \IfFileExists{parskip.sty}{%
    \usepackage{parskip}
  }{% else
    \setlength{\parindent}{0pt}
    \setlength{\parskip}{6pt plus 2pt minus 1pt}}
}{% if KOMA class
  \KOMAoptions{parskip=half}}
\makeatother
\usepackage{xcolor}
\usepackage[margin=1in]{geometry}
\usepackage{color}
\usepackage{fancyvrb}
\newcommand{\VerbBar}{|}
\newcommand{\VERB}{\Verb[commandchars=\\\{\}]}
\DefineVerbatimEnvironment{Highlighting}{Verbatim}{commandchars=\\\{\}}
% Add ',fontsize=\small' for more characters per line
\usepackage{framed}
\definecolor{shadecolor}{RGB}{248,248,248}
\newenvironment{Shaded}{\begin{snugshade}}{\end{snugshade}}
\newcommand{\AlertTok}[1]{\textcolor[rgb]{0.94,0.16,0.16}{#1}}
\newcommand{\AnnotationTok}[1]{\textcolor[rgb]{0.56,0.35,0.01}{\textbf{\textit{#1}}}}
\newcommand{\AttributeTok}[1]{\textcolor[rgb]{0.77,0.63,0.00}{#1}}
\newcommand{\BaseNTok}[1]{\textcolor[rgb]{0.00,0.00,0.81}{#1}}
\newcommand{\BuiltInTok}[1]{#1}
\newcommand{\CharTok}[1]{\textcolor[rgb]{0.31,0.60,0.02}{#1}}
\newcommand{\CommentTok}[1]{\textcolor[rgb]{0.56,0.35,0.01}{\textit{#1}}}
\newcommand{\CommentVarTok}[1]{\textcolor[rgb]{0.56,0.35,0.01}{\textbf{\textit{#1}}}}
\newcommand{\ConstantTok}[1]{\textcolor[rgb]{0.00,0.00,0.00}{#1}}
\newcommand{\ControlFlowTok}[1]{\textcolor[rgb]{0.13,0.29,0.53}{\textbf{#1}}}
\newcommand{\DataTypeTok}[1]{\textcolor[rgb]{0.13,0.29,0.53}{#1}}
\newcommand{\DecValTok}[1]{\textcolor[rgb]{0.00,0.00,0.81}{#1}}
\newcommand{\DocumentationTok}[1]{\textcolor[rgb]{0.56,0.35,0.01}{\textbf{\textit{#1}}}}
\newcommand{\ErrorTok}[1]{\textcolor[rgb]{0.64,0.00,0.00}{\textbf{#1}}}
\newcommand{\ExtensionTok}[1]{#1}
\newcommand{\FloatTok}[1]{\textcolor[rgb]{0.00,0.00,0.81}{#1}}
\newcommand{\FunctionTok}[1]{\textcolor[rgb]{0.00,0.00,0.00}{#1}}
\newcommand{\ImportTok}[1]{#1}
\newcommand{\InformationTok}[1]{\textcolor[rgb]{0.56,0.35,0.01}{\textbf{\textit{#1}}}}
\newcommand{\KeywordTok}[1]{\textcolor[rgb]{0.13,0.29,0.53}{\textbf{#1}}}
\newcommand{\NormalTok}[1]{#1}
\newcommand{\OperatorTok}[1]{\textcolor[rgb]{0.81,0.36,0.00}{\textbf{#1}}}
\newcommand{\OtherTok}[1]{\textcolor[rgb]{0.56,0.35,0.01}{#1}}
\newcommand{\PreprocessorTok}[1]{\textcolor[rgb]{0.56,0.35,0.01}{\textit{#1}}}
\newcommand{\RegionMarkerTok}[1]{#1}
\newcommand{\SpecialCharTok}[1]{\textcolor[rgb]{0.00,0.00,0.00}{#1}}
\newcommand{\SpecialStringTok}[1]{\textcolor[rgb]{0.31,0.60,0.02}{#1}}
\newcommand{\StringTok}[1]{\textcolor[rgb]{0.31,0.60,0.02}{#1}}
\newcommand{\VariableTok}[1]{\textcolor[rgb]{0.00,0.00,0.00}{#1}}
\newcommand{\VerbatimStringTok}[1]{\textcolor[rgb]{0.31,0.60,0.02}{#1}}
\newcommand{\WarningTok}[1]{\textcolor[rgb]{0.56,0.35,0.01}{\textbf{\textit{#1}}}}
\usepackage{longtable,booktabs,array}
\usepackage{calc} % for calculating minipage widths
% Correct order of tables after \paragraph or \subparagraph
\usepackage{etoolbox}
\makeatletter
\patchcmd\longtable{\par}{\if@noskipsec\mbox{}\fi\par}{}{}
\makeatother
% Allow footnotes in longtable head/foot
\IfFileExists{footnotehyper.sty}{\usepackage{footnotehyper}}{\usepackage{footnote}}
\makesavenoteenv{longtable}
\usepackage{graphicx}
\makeatletter
\def\maxwidth{\ifdim\Gin@nat@width>\linewidth\linewidth\else\Gin@nat@width\fi}
\def\maxheight{\ifdim\Gin@nat@height>\textheight\textheight\else\Gin@nat@height\fi}
\makeatother
% Scale images if necessary, so that they will not overflow the page
% margins by default, and it is still possible to overwrite the defaults
% using explicit options in \includegraphics[width, height, ...]{}
\setkeys{Gin}{width=\maxwidth,height=\maxheight,keepaspectratio}
% Set default figure placement to htbp
\makeatletter
\def\fps@figure{htbp}
\makeatother
\setlength{\emergencystretch}{3em} % prevent overfull lines
\providecommand{\tightlist}{%
  \setlength{\itemsep}{0pt}\setlength{\parskip}{0pt}}
\setcounter{secnumdepth}{-\maxdimen} % remove section numbering
\ifLuaTeX
  \usepackage{selnolig}  % disable illegal ligatures
\fi
\IfFileExists{bookmark.sty}{\usepackage{bookmark}}{\usepackage{hyperref}}
\IfFileExists{xurl.sty}{\usepackage{xurl}}{} % add URL line breaks if available
\urlstyle{same} % disable monospaced font for URLs
\hypersetup{
  pdftitle={Lab 01: Getting Started with RStudio},
  pdfauthor={Authors: Bruno Grande and Ryan Morin},
  hidelinks,
  pdfcreator={LaTeX via pandoc}}

\title{Lab 01: Getting Started with RStudio}
\author{Authors: Bruno Grande and Ryan Morin}
\date{Last updated: 2022-09-16}

\begin{document}
\maketitle

\hypertarget{quick-start}{%
\subsection{Quick start}\label{quick-start}}

\emph{Click the Knit button in the panel above to convert this file to
human-friendly HTML.}

The following code will ensure you have the necessary proxy settings in
your ssh config. It also creates a symbolic link (like a shortcut) to
the directory that contains the course files. If this is nonsense to
you, don't worry about it and just read on. \emph{If you are running
this file for the second+ time, then you need to comment out these lines
by adding \# at the start of both of them.}

\begin{Shaded}
\begin{Highlighting}[]
\CommentTok{\#mkdir \textasciitilde{}/MBB243/}
\CommentTok{\#ln {-}s /local{-}scratch/course\_files/MBB243/ \textasciitilde{}/files}
\end{Highlighting}
\end{Shaded}

The first run of the code chunk above (or the first time you knit this
document) should automatically generate a link to the folder that
contains all of the files required for our course this semester. You
don't have to worry about what this is doing, but we do need this
section in here for the first run.

\hypertarget{getting-acquainted-with-the-server-on-the-terminal}{%
\subsection{Getting acquainted with the server on the
Terminal}\label{getting-acquainted-with-the-server-on-the-terminal}}

Each RStudio user is given a home directory on the course server under
\texttt{/localhome/}. For example, the instructor's is
\texttt{/localhome/sasneddo/}. Using the Terminal window, you can find
your current working directory with the bash \texttt{pwd} command (print
working directory). The equivalent command in R itself is
\texttt{getwd()}. The table below provides a few pairs of commands for
navigating the file system in bash and R, respectively.

\begin{longtable}[]{@{}
  >{\centering\arraybackslash}p{(\columnwidth - 4\tabcolsep) * \real{0.4286}}
  >{\centering\arraybackslash}p{(\columnwidth - 4\tabcolsep) * \real{0.4286}}
  >{\raggedright\arraybackslash}p{(\columnwidth - 4\tabcolsep) * \real{0.1429}}@{}}
\toprule()
\begin{minipage}[b]{\linewidth}\centering
Bash
\end{minipage} & \begin{minipage}[b]{\linewidth}\centering
R
\end{minipage} & \begin{minipage}[b]{\linewidth}\raggedright
Effect
\end{minipage} \\
\midrule()
\endhead
\texttt{pwd} & \texttt{getwd()} & Print/return the current working
directory \\
\texttt{cd\ DIR} & \texttt{setwd("DIR")} & Change to directory specified
by DIR \\
\texttt{ls\ DIR} & \texttt{dir("DIR")} & Get a list of all the files in
DIR \\
\texttt{ls} & \texttt{dir()} & Get a list of all the files in your
working directory \\
\texttt{mkdir\ DIR} & \texttt{dir.create(DIR)} & Create a new directory
named DIR (can be full or relative) \\
\bottomrule()
\end{longtable}

\textbf{NOTE}: When you are using copy and paste to enter code into your
terminal you should NOT copy the backtick portion of the code (visible
in the Rmd file but not the html). The backticks are used only to
highlight syntax. If you copy from the html instead of the Rmd this is
not an issue.

\textbf{Task:}

Add R code and Bash code to the two empty code blocks below,
respectively, to print out your current working directory. Add a comment
to one of the code blocks that is just the full path to your home
directory similar to the examples above.

\begin{Shaded}
\begin{Highlighting}[]
\FunctionTok{getwd}\NormalTok{()}
\end{Highlighting}
\end{Shaded}

\begin{verbatim}
## [1] "/local-scratch/localhome/lprefont/lab-1-LyricTs"
\end{verbatim}

\begin{Shaded}
\begin{Highlighting}[]
\BuiltInTok{pwd}
\end{Highlighting}
\end{Shaded}

\begin{verbatim}
## /local-scratch/localhome/lprefont/lab-1-LyricTs
\end{verbatim}

\hypertarget{other-extremely-useful-bash-commands}{%
\subsection{Other extremely useful bash
commands}\label{other-extremely-useful-bash-commands}}

\begin{longtable}[]{@{}
  >{\raggedright\arraybackslash}p{(\columnwidth - 2\tabcolsep) * \real{0.5000}}
  >{\raggedright\arraybackslash}p{(\columnwidth - 2\tabcolsep) * \real{0.5000}}@{}}
\toprule()
\begin{minipage}[b]{\linewidth}\raggedright
Bash Command
\end{minipage} & \begin{minipage}[b]{\linewidth}\raggedright
Effect
\end{minipage} \\
\midrule()
\endhead
\texttt{cat\ FILE1\ FILEN} & print contents of one or more files to
STDOUT (i.e.~to the terminal) \\
\texttt{head\ FILE} & print the first few (10 by default) lines of a
file to STDOUT \\
\texttt{tail\ FILE} & print the last few (10 by default) lines of a file
to STDOUT \\
\texttt{tail\ -n\ +NUM\ FILE} & print the last lines of a file to STDOUT
starting at offset NUM \\
\texttt{rm\ FILE} & permanently delete FILE \textbf{use with
caution!} \\
\texttt{mv\ FILE\ NEWNAME} & rename or move FILE to NEWNAME \\
\texttt{less\ FILE} & open file in a scrollable text viewer (Q to
exit) \\
\texttt{wc\ FILE} & print the tabulation of characters, words and lines
to STDOUT \\
\texttt{cut\ -f\ 1,2,N\ FILE} & Take columns 1,2,N from a delimited file
and pass to STDOUT \\
\texttt{grep\ PATTERN\ FILE} & \texttt{grep} and \texttt{egrep} both
search for simple or complex patterns of text in files \\
\bottomrule()
\end{longtable}

\hypertarget{exploring-and}{%
\subsection{\texorpdfstring{Exploring
\texttt{\textbar{}},\texttt{\textgreater{}},
\texttt{\textgreater{}\textgreater{}} and
\texttt{\textless{}}}{Exploring \textbar,\textgreater, \textgreater\textgreater{} and \textless{}}}\label{exploring-and}}

Let's revisit some of the commands you saw used in class individually.
The code below runs the same command and either prints the result out or
stores it in a file. You can view the contents of the new file either in
the file explorer in Rstudio or at the Terminal (e.g.~using
\texttt{cat}).

\textbf{Task} * Duplicate the last line in this code chunk and modify it
by replacing the single \texttt{\textgreater{}} with
\texttt{\textgreater{}\textgreater{}}. What happens when you rerun it?

\begin{itemize}
\tightlist
\item
  What happens if you change the output file name (after
  \texttt{\textgreater{}\textgreater{}}) to
  \texttt{\textasciitilde{}/MBB243/out/genotype\_head.txt}? Why do you
  think that is?
\end{itemize}

\begin{Shaded}
\begin{Highlighting}[]
\FunctionTok{head} \AttributeTok{{-}n}\NormalTok{ 25 /local{-}scratch/course\_files/MBB243/Sneddon\_genotypes.txt}
\CommentTok{\# translation:}
\CommentTok{\# head: take the first N lines (default is 10 if not specified)}
\CommentTok{\# {-}n 5: specifies to take the first 5 lines instead}
\CommentTok{\# the last bit is the full path to the file we want to use as input for the head command}
\CommentTok{\# You can save the output you see by "redirecting" STDOUT to a file. }

\FunctionTok{head} \AttributeTok{{-}n}\NormalTok{ 25 /local{-}scratch/course\_files/MBB243/Sneddon\_genotypes.txt }\OperatorTok{\textgreater{}}\NormalTok{ \textasciitilde{}/MBB243/genotype\_head.txt}
\FunctionTok{head} \AttributeTok{{-}n}\NormalTok{ 25 /local{-}scratch/course\_files/MBB243/Sneddon\_genotypes.txt }\OperatorTok{\textgreater{}\textgreater{}}\NormalTok{ \textasciitilde{}/MBB243/genotype\_head.txt}
\end{Highlighting}
\end{Shaded}

\begin{verbatim}
## rs548049170  1   69869   TT
## rs13328684   1   74792   --
## rs9283150    1   565508  AA
## i713426  1   726912  AA
## rs116587930  1   727841  GG
## rs3131972    1   752721  AG
## rs12184325   1   754105  CC
## rs12567639   1   756268  AA
## rs114525117  1   759036  GG
## rs12124819   1   776546  AA
## rs12127425   1   794332  GG
## rs79373928   1   801536  TT
## rs72888853   1   815421  --
## rs7538305    1   824398  AC
## rs28444699   1   830181  AA
## i713449  1   830731  --
## rs116452738  1   834830  AG
## rs72631887   1   835092  TT
## rs28678693   1   838665  TT
## rs4970382    1   840753  CC
## rs4475691    1   846808  CC
## rs72631889   1   851390  GT
## rs7537756    1   854250  AA
## rs13302982   1   861808  GG
## rs376747791  1   863130  AA
\end{verbatim}

You should have a few lines of genotype information in a new file named
\texttt{\textasciitilde{}/MBB243/genotype\_head.txt}. Do you notice any
genotypes in there that look ``off''? Assuming we wanted to sanity check
this whole file for what genotypes exist, we can use a combination the
\texttt{cut}, \texttt{uniq} and \texttt{sort} commands (uniq is dumb and
only works properly if its input is sorted). The code chunk below does
this but uses our smaller file as input for efficiency. The result
should be all unique genotypes in that file. What did we do wrong?

\begin{Shaded}
\begin{Highlighting}[]
\FunctionTok{cut} \AttributeTok{{-}f}\NormalTok{ 4 \textasciitilde{}/MBB243/genotype\_head.txt }\KeywordTok{|} \FunctionTok{sort} \KeywordTok{|} \FunctionTok{uniq}
\CommentTok{\# to help debug this, deconstruct it into smaller pieces.}
\CommentTok{\# then try running each set of commands before the pipe, e.g. uncomment the line below and re{-}run this chunk. What is the output?  }
\CommentTok{\#This is probably easier to do at the terminal instead of in the Rmarkdown}
\FunctionTok{cut} \AttributeTok{{-}f}\NormalTok{ 4 \textasciitilde{}/MBB243/genotype\_head.txt }\KeywordTok{|} \FunctionTok{sort} 
\FunctionTok{cut} \AttributeTok{{-}f}\NormalTok{ 4 \textasciitilde{}/MBB243/genotype\_head.txt }
\end{Highlighting}
\end{Shaded}

\begin{verbatim}
## --
## AA
## AC
## AG
## CC
## GG
## GT
## TT
## --
## --
## --
## --
## --
## --
## AA
## AA
## AA
## AA
## AA
## AA
## AA
## AA
## AA
## AA
## AA
## AA
## AA
## AA
## AC
## AC
## AG
## AG
## AG
## AG
## CC
## CC
## CC
## CC
## CC
## CC
## GG
## GG
## GG
## GG
## GG
## GG
## GG
## GG
## GT
## GT
## TT
## TT
## TT
## TT
## TT
## TT
## TT
## TT
## TT
## --
## AA
## AA
## GG
## AG
## CC
## AA
## GG
## AA
## GG
## TT
## --
## AC
## AA
## --
## AG
## TT
## TT
## CC
## CC
## GT
## AA
## GG
## AA
## TT
## --
## AA
## AA
## GG
## AG
## CC
## AA
## GG
## AA
## GG
## TT
## --
## AC
## AA
## --
## AG
## TT
## TT
## CC
## CC
## GT
## AA
## GG
## AA
\end{verbatim}

\hypertarget{setting-the-value-of-a-variable}{%
\subsection{Setting the value of a
variable}\label{setting-the-value-of-a-variable}}

Basic variables store a single value that can be changed over time
within your program. Creating (or ``declaring'' a new variable is
usually coupled with setting it's value. It is common to use a variable
to capture or store the output of some other code or a function.
Anything that is printed to your screen in Rstudio or shown in the
output of your markdown can instead be stored in a variable. Why might
we want to do this? Here's an example that we'll use later. The data
files that we will use in each lab are all being put in one shared
directory on this server:

\texttt{/local-scratch/course\_files/MBB243/}

Do you want to type that whole directory or paste it into your code
every time you need to refer to a file there? Me neither! We can instead
create a variable that stores that information and use it as a shortcut
to represent that path. In bash, here's how we can create a variable
named \texttt{DATA} and store this information in it. The next few lines
show how it can be put to use various ways. This is a bit irrelevant for
this directory since we created a symbolic link to it above and that
acts as a shortcut but it isn't as versatile. For example, from any
working directory, you can use \$DATA to refer to that file path but you
would need to be in your home directory to use the symbolic link
(\texttt{files}) or specify the path to that link
(\texttt{\textasciitilde{}/files}) if you are in another directory.

\begin{Shaded}
\begin{Highlighting}[]
\VariableTok{DATA}\OperatorTok{=}\StringTok{"/local{-}scratch/course\_files/MBB243/"} 
\CommentTok{\# }\AlertTok{NOTE}\CommentTok{: if you aren\textquotesingle{}t using the course server you will need to copy the contents of this directory to whatever computer you are running Rstudio on.}

\CommentTok{\# print the contents of our new variable with the echo command}
\CommentTok{\# IMPORTANT: when retrieving the contents of a variable we need to preface it with $}
\BuiltInTok{echo}\NormalTok{ data path is }\VariableTok{$DATA}
\BuiltInTok{echo}\NormalTok{ another way to refer to it is \textasciitilde{}/files}
\CommentTok{\# how about using it in combination with ls to view what\textquotesingle{}s in the directory?}
\BuiltInTok{echo}\NormalTok{ contents of the directory:}
\FunctionTok{ls} \AttributeTok{{-}l} \VariableTok{$DATA}
\BuiltInTok{echo}\NormalTok{ contents of symbolic link to directory:}
\FunctionTok{ls} \AttributeTok{{-}l}\NormalTok{ \textasciitilde{}/files/}
\end{Highlighting}
\end{Shaded}

\begin{verbatim}
## data path is /local-scratch/course_files/MBB243/
## another way to refer to it is /localhome/lprefont/files
## contents of the directory:
## total 3723784
## -rw-r--r--. 1 rdmorin  domain users         62 Jan 16  2022 bash_profile_template.txt
## drwxr-xr-x. 3 rdmorin  domain users         32 Aug  8 13:45 bin
## -rw-r--r--. 1 rdmorin  domain users   12421120 Sep  2  2011 chromFaMasked.tar
## -rw-r--r--. 1 rdmorin  domain users        125 Jan 10  2022 condarc_template.txt
## -rw-r--r--. 1 rdmorin  domain users     295777 Sep  9  2020 covid19_cases_bc.csv
## -rw-r--r--. 1 rdmorin  domain users        180 Jul  2  2020 covid19_cases_canada_monthly.csv
## -rw-r--r--. 1 rdmorin  domain users     129652 Sep  9  2020 covid19_cases_provinces.csv
## -rw-r--r--. 1 rdmorin  domain users      18984 Sep  9  2020 covid19_cases_provinces_weekly.csv
## -rw-r--r--. 1 rdmorin  domain users    2013935 Sep 28  2020 covid19_cases_worldwide.csv
## -rw-r--r--. 1 rdmorin  domain users      56053 Sep 28  2020 covid19_cases_worldwide_monthly.csv
## -rw-r--r--. 1 rdmorin  domain users      75219 Mar 11  2022 covid19_cases_worldwide_monthly.updated.csv
## -rw-r--r--. 1 rdmorin  domain users        101 Jan  6  2022 cutsite_ref.fa
## -rw-r--r--. 1 rdmorin  domain users  130014152 Mar 18 05:44 ensembl_genes_hg38.tsv
## -rw-r--r--. 1 rdmorin  domain users          0 Jan 14  2022 ENST00000003084
## lrwxrwxrwx. 1 rdmorin  domain users          5 Mar 11  2022 files -> files
## -rw-r--r--. 1 rdmorin  domain users    1249196 Apr  1 03:00 gambl_capture_mutmat_coo.tsv
## -rw-r--r--. 1 rdmorin  domain users     711215 Jan  9  2022 gene_id.sorted.txt
## -rw-r--r--. 1 rdmorin  domain users    2194547 Jan  8  2022 gene_id.txt
## -rw-r--r--. 1 rdmorin  domain users   24780911 Mar  8  2022 genome_Ashley_Bodily.txt
## -rw-r--r--. 1 rdmorin  domain users        126 Jan  6  2022 genotype_counts.txt
## -rw-r--r--. 1 rdmorin  domain users       8556 Sep 28  2020 get_data.R
## -rw-r--r--. 1 rdmorin  domain users     706464 Jan 18  2022 GSE125966_annotated.csv
## -rw-r--r--. 1 rdmorin  domain users   48430130 Jan  7  2022 GSE125966.csv
## -rw-r--r--. 1 rdmorin  domain users      54594 Mar  6  2022 GSE125966_GOYA_mini.csv
## -rw-r--r--. 1 rdmorin  domain users  268232379 Jan  7  2022 GSE125966_GOYA_stranded_log2CPM.csv
## -rw-r--r--. 1 rdmorin  domain users 3144230986 Jan  5  2022 hg38.fa
## -rw-r--r--. 1 rdmorin  domain users       7804 Jan  6  2022 hg38.fa.fai
## -rw-r--r--. 1 rdmorin  domain users    2050717 Feb 17  2022 human_CDS_chr7.fa.gz
## drwxr-xr-x. 2 rdmorin  domain users     425984 Jan 14  2022 human_genes_chr7
## -rw-r--r--. 1 rdmorin  domain users   20821185 Jan 14  2022 human_genes_chr7.fa
## -rw-r--r--. 1 rdmorin  domain users    7472874 Jan 14  2022 human_genes_chr7.tar.gz
## lrwxrwxrwx. 1 rdmorin  domain users         35 Jan  8  2022 MBB243 -> /local-scratch/course_files/MBB243/
## drwxr-xr-x. 2 rdmorin  domain users         10 Apr  8 09:24 md
## -rwxr-xr-x. 1 rdmorin  domain users   66709754 Jan  6  2022 miniconda_installer.sh
## -rw-------. 1 rdmorin  domain users   24808135 Jan  5  2022 Morin_genotypes.txt
## -rw-r--r--. 1 rdmorin  domain users    8682096 Jan 14  2022 mouse_cds_chr7.fa.txt
## -rw-r--r--. 1 rdmorin  domain users    2563618 Jul 15  2020 pannets_expr_array.csv.gz
## -rw-r--r--. 1 rdmorin  domain users    2036945 Jul 14  2020 pannets_expr_rnaseq.csv.gz
## -rw-r--r--. 1 rdmorin  domain users       4886 Jul 14  2020 pannets_metadata.csv
## -rw-r--r--. 1 rdmorin  domain users        302 Jul  2  2020 population_canada.csv
## -rw-r--r--. 1 rdmorin  domain users         86 Jun 24  2020 README.md
## -rw-r--r--. 1 sasneddo domain users   16838942 Aug  8 16:13 Sneddon_genotypes.txt
## -rw-r--r--. 1 rdmorin  domain users      12331 Feb  2  2022 some_human_genes.fa
## -rw-r--r--. 1 rdmorin  domain users         67 Jan  8  2022 ssh_config_template.txt
## -rw-r--r--. 1 rdmorin  domain users        131 Jan  6  2022 test.txt
## -rw-r--r--. 1 rdmorin  domain users      34542 Feb  9  2022 words.txt
## drwxr-xr-x. 2 rdmorin  domain users       4096 Feb  3  2022 yeast_chromosomes
## -rw-r--r--. 1 rdmorin  domain users   12400389 Feb  3  2022 yeast_genome.fa
## -rw-r--r--. 1 rdmorin  domain users   12400379 Feb  3  2022 yeast_genome.fa~
## contents of symbolic link to directory:
## total 3723784
## -rw-r--r--. 1 rdmorin  domain users         62 Jan 16  2022 bash_profile_template.txt
## drwxr-xr-x. 3 rdmorin  domain users         32 Aug  8 13:45 bin
## -rw-r--r--. 1 rdmorin  domain users   12421120 Sep  2  2011 chromFaMasked.tar
## -rw-r--r--. 1 rdmorin  domain users        125 Jan 10  2022 condarc_template.txt
## -rw-r--r--. 1 rdmorin  domain users     295777 Sep  9  2020 covid19_cases_bc.csv
## -rw-r--r--. 1 rdmorin  domain users        180 Jul  2  2020 covid19_cases_canada_monthly.csv
## -rw-r--r--. 1 rdmorin  domain users     129652 Sep  9  2020 covid19_cases_provinces.csv
## -rw-r--r--. 1 rdmorin  domain users      18984 Sep  9  2020 covid19_cases_provinces_weekly.csv
## -rw-r--r--. 1 rdmorin  domain users    2013935 Sep 28  2020 covid19_cases_worldwide.csv
## -rw-r--r--. 1 rdmorin  domain users      56053 Sep 28  2020 covid19_cases_worldwide_monthly.csv
## -rw-r--r--. 1 rdmorin  domain users      75219 Mar 11  2022 covid19_cases_worldwide_monthly.updated.csv
## -rw-r--r--. 1 rdmorin  domain users        101 Jan  6  2022 cutsite_ref.fa
## -rw-r--r--. 1 rdmorin  domain users  130014152 Mar 18 05:44 ensembl_genes_hg38.tsv
## -rw-r--r--. 1 rdmorin  domain users          0 Jan 14  2022 ENST00000003084
## lrwxrwxrwx. 1 rdmorin  domain users          5 Mar 11  2022 files -> files
## -rw-r--r--. 1 rdmorin  domain users    1249196 Apr  1 03:00 gambl_capture_mutmat_coo.tsv
## -rw-r--r--. 1 rdmorin  domain users     711215 Jan  9  2022 gene_id.sorted.txt
## -rw-r--r--. 1 rdmorin  domain users    2194547 Jan  8  2022 gene_id.txt
## -rw-r--r--. 1 rdmorin  domain users   24780911 Mar  8  2022 genome_Ashley_Bodily.txt
## -rw-r--r--. 1 rdmorin  domain users        126 Jan  6  2022 genotype_counts.txt
## -rw-r--r--. 1 rdmorin  domain users       8556 Sep 28  2020 get_data.R
## -rw-r--r--. 1 rdmorin  domain users     706464 Jan 18  2022 GSE125966_annotated.csv
## -rw-r--r--. 1 rdmorin  domain users   48430130 Jan  7  2022 GSE125966.csv
## -rw-r--r--. 1 rdmorin  domain users      54594 Mar  6  2022 GSE125966_GOYA_mini.csv
## -rw-r--r--. 1 rdmorin  domain users  268232379 Jan  7  2022 GSE125966_GOYA_stranded_log2CPM.csv
## -rw-r--r--. 1 rdmorin  domain users 3144230986 Jan  5  2022 hg38.fa
## -rw-r--r--. 1 rdmorin  domain users       7804 Jan  6  2022 hg38.fa.fai
## -rw-r--r--. 1 rdmorin  domain users    2050717 Feb 17  2022 human_CDS_chr7.fa.gz
## drwxr-xr-x. 2 rdmorin  domain users     425984 Jan 14  2022 human_genes_chr7
## -rw-r--r--. 1 rdmorin  domain users   20821185 Jan 14  2022 human_genes_chr7.fa
## -rw-r--r--. 1 rdmorin  domain users    7472874 Jan 14  2022 human_genes_chr7.tar.gz
## lrwxrwxrwx. 1 rdmorin  domain users         35 Jan  8  2022 MBB243 -> /local-scratch/course_files/MBB243/
## drwxr-xr-x. 2 rdmorin  domain users         10 Apr  8 09:24 md
## -rwxr-xr-x. 1 rdmorin  domain users   66709754 Jan  6  2022 miniconda_installer.sh
## -rw-------. 1 rdmorin  domain users   24808135 Jan  5  2022 Morin_genotypes.txt
## -rw-r--r--. 1 rdmorin  domain users    8682096 Jan 14  2022 mouse_cds_chr7.fa.txt
## -rw-r--r--. 1 rdmorin  domain users    2563618 Jul 15  2020 pannets_expr_array.csv.gz
## -rw-r--r--. 1 rdmorin  domain users    2036945 Jul 14  2020 pannets_expr_rnaseq.csv.gz
## -rw-r--r--. 1 rdmorin  domain users       4886 Jul 14  2020 pannets_metadata.csv
## -rw-r--r--. 1 rdmorin  domain users        302 Jul  2  2020 population_canada.csv
## -rw-r--r--. 1 rdmorin  domain users         86 Jun 24  2020 README.md
## -rw-r--r--. 1 sasneddo domain users   16838942 Aug  8 16:13 Sneddon_genotypes.txt
## -rw-r--r--. 1 rdmorin  domain users      12331 Feb  2  2022 some_human_genes.fa
## -rw-r--r--. 1 rdmorin  domain users         67 Jan  8  2022 ssh_config_template.txt
## -rw-r--r--. 1 rdmorin  domain users        131 Jan  6  2022 test.txt
## -rw-r--r--. 1 rdmorin  domain users      34542 Feb  9  2022 words.txt
## drwxr-xr-x. 2 rdmorin  domain users       4096 Feb  3  2022 yeast_chromosomes
## -rw-r--r--. 1 rdmorin  domain users   12400389 Feb  3  2022 yeast_genome.fa
## -rw-r--r--. 1 rdmorin  domain users   12400379 Feb  3  2022 yeast_genome.fa~
\end{verbatim}

\hypertarget{setting-up-bash-preferences-and-managing-custom-software}{%
\subsection{Setting up bash preferences and managing custom
software}\label{setting-up-bash-preferences-and-managing-custom-software}}

Many things you do in the bash terminal are ``one-off'' commands that
wouldn't make sense to run in a markdown document. For example, you can
create, destroy, manipulate or rename files from bash and these changes
will exist outside the sandboxed environment. For example, the first
time you run this code chunk you will create a new file and add one line
of text to it. In most scenarios, you are much more likely to use bash
directly in an interactive shell session. RStudio gives you a terminal
session where you can run bash commands interactively and navigate the
file system.

\textbf{Task:}

Run the following code chunk once then use the RStudio terminal (not the
Console) to find and view the contents of that file. If you have already
``knit'' the document then the code has been run. The \texttt{cat}
command will dump the contents of any file to your terminal and is a
convenient way to look at the contents of a file. For larger files, it's
better to use either \texttt{head} to view the first few lines or
\texttt{tail} to view the last few. What changes in the file if you
manually run the code chunk a few more times?

\begin{Shaded}
\begin{Highlighting}[]
\BuiltInTok{echo} \StringTok{"Hello terminal!"} \OperatorTok{\textgreater{}\textgreater{}}\NormalTok{ EMPTY.txt}
\BuiltInTok{echo} \VariableTok{$(}\FunctionTok{date}\VariableTok{)} \OperatorTok{\textgreater{}\textgreater{}}\NormalTok{ EMPTY.txt}
\FunctionTok{cat}\NormalTok{ EMPTY.txt}
\CommentTok{\# The redirect operators \textgreater{} and \textgreater{}\textgreater{} are related but have different effects on existing files. It\textquotesingle{}s important that you know the difference. }
\CommentTok{\# If you aren\textquotesingle{}t clear, go back to the earlier portion of this document that introduces it and/or ask the TA or instructor. }
\end{Highlighting}
\end{Shaded}

\begin{verbatim}
## Hello terminal!
## Fri Sep 9 14:22:16 PDT 2022
## Hello terminal!
## Fri Sep 9 14:41:28 PDT 2022
## Hello terminal!
## Fri Sep 9 14:41:33 PDT 2022
## Hello terminal!
## Mon Sep 12 14:35:54 PDT 2022
## Hello terminal!
## Mon Sep 12 14:39:07 PDT 2022
## Hello terminal!
## Mon Sep 12 14:39:30 PDT 2022
## Hello terminal!
## Mon Sep 12 14:39:34 PDT 2022
## Hello terminal!
## Mon Sep 12 14:40:27 PDT 2022
## Hello terminal!
## Mon Sep 12 14:40:31 PDT 2022
## Hello terminal!
## Mon Sep 12 14:40:42 PDT 2022
## Hello terminal!
## Tue Sep 13 22:35:54 PDT 2022
## Hello terminal!
## Tue Sep 13 22:44:09 PDT 2022
## Hello terminal!
## Tue Sep 13 22:44:12 PDT 2022
## Hello terminal!
## Tue Sep 13 22:47:17 PDT 2022
## Hello terminal!
## Tue Sep 13 23:36:32 PDT 2022
## Hello terminal!
## Tue Sep 13 23:44:04 PDT 2022
## Hello terminal!
## Tue Sep 13 23:47:18 PDT 2022
## Hello terminal!
## Tue Sep 13 23:47:24 PDT 2022
## Hello terminal!
## Wed Sep 14 00:10:54 PDT 2022
## Hello terminal!
## Wed Sep 14 13:23:22 PDT 2022
## Hello terminal!
## Wed Sep 14 14:08:28 PDT 2022
## Hello terminal!
## Wed Sep 14 14:09:16 PDT 2022
## Hello terminal!
## Wed Sep 14 14:11:33 PDT 2022
## Hello terminal!
## Wed Sep 14 14:40:01 PDT 2022
## Hello terminal!
## Wed Sep 14 14:41:18 PDT 2022
## Hello terminal!
## Fri Sep 16 12:46:53 PDT 2022
\end{verbatim}

\textbf{Task:}

Make a copy of the code chunk above and put it below this text in your
version of the Rmd file. Modify the code so it instead creates a file
with this exact name: \texttt{\textasciitilde{}/.bash\_profile}.
\emph{General hint}: when you can, use copy/paste so you don't have to
re-type a command or file name. You also need to change the code chunk
to add a single line to the new file. The first line of text should be:

\texttt{source\ \textasciitilde{}/.bashrc}

The second line should be:

\texttt{DATA=/local-scratch/course\_files/MBB243/}

Be careful here with the distinction between \texttt{\textgreater{}} and
\texttt{\textgreater{}\textgreater{}}. The code should only ``append''
lines to the file if/when necessary (i.e.~when re-run you will not get
additional lines in the file). You can check that your code is working
by running the chunk more than once and then verifying the contents of
the file. If you are so inclined, you can even add a line to your code
chunk that will report the current number of lines in the file each time
the code chunk is run. Give it a try if you're feeling ambitious! By the
end of this lab you will need to have completed this such that you have
a bash\_profile file with those two lines in it. See the instructor or
TA for help if you don't think you succeeded. The easist way to confirm
is to run this command in your terminal (not in Rmarkdown) and looking
at the contents:

\texttt{cat\ \textasciitilde{}/.bash\_profile}

\begin{Shaded}
\begin{Highlighting}[]
\BuiltInTok{echo} \StringTok{"source \textasciitilde{}/.bashrc"} \OperatorTok{\textgreater{}}\NormalTok{ \textasciitilde{}/.bash\_profile}
\BuiltInTok{echo} \StringTok{"DATA=/local{-}scratch/course\_files/MBB243/"} \OperatorTok{\textgreater{}\textgreater{}}\NormalTok{ \textasciitilde{}/.bash\_profile}
\FunctionTok{cat}\NormalTok{ \textasciitilde{}/.bash\_profile}
\end{Highlighting}
\end{Shaded}

\begin{verbatim}
## source ~/.bashrc
## DATA=/local-scratch/course_files/MBB243/
\end{verbatim}

\hypertarget{setting-a-variable-in-r}{%
\subsection{Setting a variable in R}\label{setting-a-variable-in-r}}

All programming languages has variables. R has a much more complex (and
useful) set of variable types. Here we will just repeat what we learned
in bash to see how the syntax looks. In this case, it turns out that
using the bash syntax does the same thing but that's not always the
case. The example below shows two ways to accomplish the same thing in R
with the second example illustrating the preferred syntax to use in R
(\texttt{\textless{}-} instead of \texttt{=}).

\begin{Shaded}
\begin{Highlighting}[]
\NormalTok{DATA}\OtherTok{=}\StringTok{"/local{-}scratch/course\_files/MBB243/"}
\FunctionTok{print}\NormalTok{(DATA)}
\end{Highlighting}
\end{Shaded}

\begin{verbatim}
## [1] "/local-scratch/course_files/MBB243/"
\end{verbatim}

\begin{Shaded}
\begin{Highlighting}[]
\CommentTok{\#do the same thing in a preferred syntax}
\NormalTok{DATA }\OtherTok{\textless{}{-}} \StringTok{"/local{-}scratch/course\_files/MBB243/"}
\FunctionTok{print}\NormalTok{(DATA)}
\end{Highlighting}
\end{Shaded}

\begin{verbatim}
## [1] "/local-scratch/course_files/MBB243/"
\end{verbatim}

\begin{Shaded}
\begin{Highlighting}[]
\CommentTok{\#get directory contents}
\FunctionTok{dir}\NormalTok{(DATA)}
\end{Highlighting}
\end{Shaded}

\begin{verbatim}
##  [1] "bash_profile_template.txt"                  
##  [2] "bin"                                        
##  [3] "chromFaMasked.tar"                          
##  [4] "condarc_template.txt"                       
##  [5] "covid19_cases_bc.csv"                       
##  [6] "covid19_cases_canada_monthly.csv"           
##  [7] "covid19_cases_provinces_weekly.csv"         
##  [8] "covid19_cases_provinces.csv"                
##  [9] "covid19_cases_worldwide_monthly.csv"        
## [10] "covid19_cases_worldwide_monthly.updated.csv"
## [11] "covid19_cases_worldwide.csv"                
## [12] "cutsite_ref.fa"                             
## [13] "ensembl_genes_hg38.tsv"                     
## [14] "ENST00000003084"                            
## [15] "files"                                      
## [16] "gambl_capture_mutmat_coo.tsv"               
## [17] "gene_id.sorted.txt"                         
## [18] "gene_id.txt"                                
## [19] "genome_Ashley_Bodily.txt"                   
## [20] "genotype_counts.txt"                        
## [21] "get_data.R"                                 
## [22] "GSE125966_annotated.csv"                    
## [23] "GSE125966_GOYA_mini.csv"                    
## [24] "GSE125966_GOYA_stranded_log2CPM.csv"        
## [25] "GSE125966.csv"                              
## [26] "hg38.fa"                                    
## [27] "hg38.fa.fai"                                
## [28] "human_CDS_chr7.fa.gz"                       
## [29] "human_genes_chr7"                           
## [30] "human_genes_chr7.fa"                        
## [31] "human_genes_chr7.tar.gz"                    
## [32] "MBB243"                                     
## [33] "md"                                         
## [34] "miniconda_installer.sh"                     
## [35] "Morin_genotypes.txt"                        
## [36] "mouse_cds_chr7.fa.txt"                      
## [37] "pannets_expr_array.csv.gz"                  
## [38] "pannets_expr_rnaseq.csv.gz"                 
## [39] "pannets_metadata.csv"                       
## [40] "population_canada.csv"                      
## [41] "README.md"                                  
## [42] "Sneddon_genotypes.txt"                      
## [43] "some_human_genes.fa"                        
## [44] "ssh_config_template.txt"                    
## [45] "test.txt"                                   
## [46] "words.txt"                                  
## [47] "yeast_chromosomes"                          
## [48] "yeast_genome.fa"                            
## [49] "yeast_genome.fa~"
\end{verbatim}

\begin{Shaded}
\begin{Highlighting}[]
\CommentTok{\#store directory contents in a new variable}
\NormalTok{data\_list }\OtherTok{=} \FunctionTok{dir}\NormalTok{(DATA)}
\end{Highlighting}
\end{Shaded}

\hypertarget{configuring-your-terminal-to-run-custom-software}{%
\subsection{Configuring your terminal to run custom
software}\label{configuring-your-terminal-to-run-custom-software}}

There are some things we need to set up for each user that are great
examples of bash code that we want to be reproducible. One of them is to
ensure any software you install on the server will always be available.
Unlike some operating systems, Linux allows individual users to install
software in any location they have the appropriate permissions. Users
can install different versions/flavours of the same software on one
computer. For any such software to be run, the shell needs to know where
to look for it. When we open a new terminal session (i.e.~bash is
launched) the contents of a few files are automatically interpreted.
These configuration files are often named with a \texttt{.} character
before the filename, which is a way to ``hide'' files from normal
directory listing. For example, today we will work with
\texttt{.bash\_profile}, \texttt{.bashrc} and \texttt{.condarc}. To see
the hidden files in the Files pane you can select ``show hidden files''
under the ``More'' menu. If you did the task above properly, you should
now have a .bashrc and a .bash\_profile file.

\hypertarget{installing-miniconda-and-snakemake}{%
\subsection{Installing miniconda and
snakemake}\label{installing-miniconda-and-snakemake}}

Some of the software we will need later in the semester isn't installed
on the server. Here we will install the miniconda package, which makes
the process of installing and maintaining multiple versions of other
tools relatively painless. For your bash session to be aware of
software, the location of it's executable needs to be known to the
environment. Bash has special variables that keep track of various
important configuration. The location of available software is stored in
the environment variable known as PATH. Notably, you have a ton of
environment variables that are convenient shortcuts. Try adding some
code below to print the contents of other environment variables such as
HOME, HOSTNAME, and USER.

\begin{Shaded}
\begin{Highlighting}[]
\BuiltInTok{echo} \VariableTok{$PATH}
\end{Highlighting}
\end{Shaded}

\begin{verbatim}
## /localhome/lprefont/miniconda3/bin:/localhome/lprefont/miniconda3/condabin:/localhome/lprefont/.local/bin:/localhome/lprefont/bin:/local-scratch/localhome/lprefont/miniconda3/bin:/localhome/lprefont/miniconda3/condabin:/usr/local/sbin:/usr/local/bin:/usr/sbin:/usr/bin:/usr/lib/rstudio-server/bin/quarto/bin:/usr/lib/rstudio-server/bin/postback:/usr/lib/rstudio-server/bin/postback
\end{verbatim}

\textbf{Task}

In the Terminal you need to run the script miniconda\_installer.sh. This
will launch an interactive process so you will have to run it directly
in the terminal not in a code chunk. To run it, simply enter the full
path to the installer script
\texttt{/local-scratch/course\_files/MBB243/miniconda\_installer.sh} in
the Terminal and hit ENTER. You must respond to the on-screen prompts.
You have to hit the ENTER key and (theoretically) you will read the
EULA. Then you will need to type ``yes'' and hit ENTER again. Hit ENTER
again at each prompt. Don't change any of the default options.
\emph{IMPORTANT}: To test that your installation worked you will need to
reload your bash environment. This can always be done by running
\texttt{source\ \textasciitilde{}/.bashrc}. If you did this step and the
previous steps properly, your prompt should now look like this:

\texttt{(base)\ {[}yourname@mbb-bioinf\ yourname{]}\$}

\hypertarget{executing-python-in-rstudio}{%
\subsection{Executing Python in
RStudio}\label{executing-python-in-rstudio}}

The R language has many benefits for the analysis of data types commonly
produced in the life sciences. Python is another popular programming
language that will be covered in this course. For our labs, we will
configure RStudio to evaluate code blocks in either language. This first
requires the installation and loading of the \texttt{reticulate}
package. If you are running this on the lab server it should already be
installed. The following code should install (if necessary) and load the
package. The third line configures the library to use a specific python
interpreter. \textbf{you shouldn't have to uncomment the first line if
the server is set up properly}

\begin{Shaded}
\begin{Highlighting}[]
\CommentTok{\#install.packages("reticulate")}
\FunctionTok{library}\NormalTok{(reticulate)}
\FunctionTok{use\_python}\NormalTok{(}\StringTok{"\textasciitilde{}/miniconda3/bin/python"}\NormalTok{)}
\end{Highlighting}
\end{Shaded}

When we want to use Python instead of R we simply need to change the
first part of the code block to specify the language. Although you can't
easily tell from the example below, the code is being interpreted as
Python code. The two examples below show you how switching between the
two interpreters can yield different results even if the syntax is
``valid'' for both.

\begin{Shaded}
\begin{Highlighting}[]
\NormalTok{this\_string }\OperatorTok{=} \StringTok{"ATT"} \OperatorTok{+} \StringTok{"TAG"}
\NormalTok{this\_string}
\end{Highlighting}
\end{Shaded}

\begin{verbatim}
## 'ATTTAG'
\end{verbatim}

\begin{Shaded}
\begin{Highlighting}[]
\NormalTok{this\_string }\OtherTok{=} \StringTok{"ATT"} \SpecialCharTok{+} \StringTok{"TAG"}
\NormalTok{this\_string}
\end{Highlighting}
\end{Shaded}

The second code block will throw an error if evaluated. Try running the
block or change it to \texttt{eval=TRUE} then knit the document.

\hypertarget{installing-biopython}{%
\subsection{Installing BioPython}\label{installing-biopython}}

Conda can be used to install both command-line tools and Python
libraries. Run the following command at your Terminal and respond to any
prompts.

\begin{verbatim}
conda install -c conda-forge biopython
\end{verbatim}

\hypertarget{testing-biopython}{%
\subsection{Testing BioPython}\label{testing-biopython}}

To run Python code we always need to ensure we've loaded the
\texttt{reticulate} R package and specified the python interpreter we
want to use. The code below also configures reticulate to use the
miniconda version you theoretically just updated. To actually run this
part of the code you will first need to have completed the earlier
tasks.

\textbf{Task} Set eval=TRUE in the two chunks below and run them (in
order!).

\begin{Shaded}
\begin{Highlighting}[]
\FunctionTok{library}\NormalTok{(reticulate)}
\FunctionTok{use\_python}\NormalTok{(}\StringTok{"\textasciitilde{}/miniconda3/bin/python"}\NormalTok{)}
\FunctionTok{use\_condaenv}\NormalTok{(}\StringTok{"base"}\NormalTok{)}
\end{Highlighting}
\end{Shaded}

The next code block simply prints the path to the python executable that
is used and then loads and tests some functionality provided by the
BioPython library.

\begin{Shaded}
\begin{Highlighting}[]
\ImportTok{import}\NormalTok{ sys}
\NormalTok{sys.executable}
\end{Highlighting}
\end{Shaded}

\begin{verbatim}
## '/local-scratch/localhome/lprefont/miniconda3/bin/python'
\end{verbatim}

\begin{Shaded}
\begin{Highlighting}[]
\ImportTok{from}\NormalTok{ Bio.Seq }\ImportTok{import}\NormalTok{ Seq}
\NormalTok{my\_seq }\OperatorTok{=}\NormalTok{ Seq(}\StringTok{"AGTACACTGGT"}\NormalTok{)}
\NormalTok{complemented\_seq }\OperatorTok{=}\NormalTok{ my\_seq.complement()}
\BuiltInTok{str}\NormalTok{(complemented\_seq)}
\end{Highlighting}
\end{Shaded}

\begin{verbatim}
## 'TCATGTGACCA'
\end{verbatim}

\hypertarget{installing-more-utilities}{%
\subsection{Installing more utilities}\label{installing-more-utilities}}

We also need the program \texttt{samtools} installed to perform our next
(optional) task. It will be needed later in the semester so please
install it now even if you opt out of the last task.

\textbf{Task} Run this command in your terminal and make sure it
completes successfully. Test it by simply entering \texttt{samtools} in
your terminal (followed by ).

\begin{verbatim}
conda install -c bioconda samtools
\end{verbatim}

\textbf{Optional Task}

The human reference genome in fasta format can be found in the lab data
directory, specifically
\texttt{/local-scratch/course\_files/MBB243/hg38.fa}. Using what you
learned in lecture, construct a command that uses samtools to extract
the sequence from this region: \texttt{chr14:65102304-65102339} and
store the sequence as a variable. Note that we don't want to store the
``header'' (starts with \texttt{\textgreater{}}). Construct this code in
the code chunk below. Your code chunk should end with an \texttt{echo}
command that prints the contents of your new variable.

Think about how you might make this code output to a file instead of
just to the terminal. How might this be useful?

\hypertarget{lab-completion-checklist}{%
\subsection{Lab Completion Checklist}\label{lab-completion-checklist}}

\textbf{Task} At the end of every lab, replace TODO with DONE to the
left of each task to indicate you've completed it

\begin{longtable}[]{@{}
  >{\raggedright\arraybackslash}p{(\columnwidth - 2\tabcolsep) * \real{0.5000}}
  >{\raggedright\arraybackslash}p{(\columnwidth - 2\tabcolsep) * \real{0.5000}}@{}}
\toprule()
\begin{minipage}[b]{\linewidth}\raggedright
status
\end{minipage} & \begin{minipage}[b]{\linewidth}\raggedright
task
\end{minipage} \\
\midrule()
\endhead
DONE & This file has been knit \\
DONE & My .bashrc file exists \\
DONE & Miniconda is installed \\
DONE & I was able to install both Bioconda and samtools using conda \\
TODO & The last few code chunks that run Python code run without errors
when I knit this document \\
\bottomrule()
\end{longtable}

\end{document}
